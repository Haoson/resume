%%%%%%%%%%%%%%%%%%%%%%%%%%%%%%%%%%%%%%%%%
% Two Column One Page Curriculum Vitae
% LaTeX Template
% Version 1.1 (24/1/13)
%
% This template has been downloaded from:
% http://www.LaTeXTemplates.com
%
% Original author:
% Alessandro (The CV Inn)
%
% IMPORTANT: THIS TEMPLATE NEEDS TO BE COMPILED WITH XeLaTeX
%
% This template uses several fonts not included with Windows/Linux by
% default. If you get compilation errors saying a font is missing, find the line
% on which the font is used and either change it to a font included with your
% operating system or comment the line out to use the default font.
% 
%%%%%%%%%%%%%%%%%%%%%%%%%%%%%%%%%%%%%%%%%

%----------------------------------------------------------------------------------------
%	PACKAGES AND OTHER DOCUMENT CONFIGURATIONS
%----------------------------------------------------------------------------------------

\documentclass[10pt]{article} % Font size - 10pt, 11pt or 12pt

\usepackage[hmargin=1.25cm, vmargin=1.5cm]{geometry} % Document margins

\usepackage{marvosym} % Required for symbols in the colored box
\usepackage{ifsym} % Required for symbols in the colored box

\usepackage[usenames,dvipsnames]{xcolor} % Allows the definition of hex colors

% Fonts and tweaks for XeLaTeX
\usepackage{fontspec,xltxtra,xunicode}
\defaultfontfeatures{Mapping=tex-text}
\usepackage{fontspec,xltxtra,xunicode}
\usepackage{latexsym}
\usepackage{xeCJK}

\defaultfontfeatures{Mapping=tex-text}
\setCJKmainfont[BoldFont=SimHei,ItalicFont=KaiTi]{SimSun}
\setmainfont{Palatino Linotype}
%\setmonofont[Scale=MatchLowercase]{Andale Mono}

% Colors for links, text and headings
\usepackage{hyperref}
\definecolor{linkcolor}{HTML}{506266} % Blue-gray color for links
\definecolor{shade}{HTML}{DFDFDF} % Peach color for the contact information box
\definecolor{text1}{HTML}{2b2b2b} % Main document font color, off-black
\definecolor{headings}{HTML}{701112} % Dark red color for headings
% Other color palettes: shade=B9D7D9 and linkcolor=A40000; shade=D4D7FE and linkcolor=FF0080

\hypersetup{colorlinks,breaklinks, urlcolor=linkcolor, linkcolor=linkcolor} % Set up links and colors

\usepackage{fancyhdr}
\pagestyle{fancy}
\fancyhf{}
% Headers and footers can be added with the \lhead{} \rhead{} \lfoot{} \rfoot{} commands
% Example footer:
%\rfoot{\color{headings} {\sffamily Last update: \today}. Typeset with Xe\LaTeX}

\renewcommand{\headrulewidth}{0pt} % Get rid of the default rule in the header

\usepackage{titlesec} % Allows creating custom \section's

% Format of the section titles
\titleformat{\section}{\color{headings}
\scshape\Large\raggedright}{}{0em}{}[\color{black}\titlerule]

\titlespacing{\section}{0pt}{0pt}{5pt} % Spacing around titles

\begin{document}

\color{text1} % Sets the default text color for the whole document to the color defined as 'text1'

%----------------------------------------------------------------------------------------
%	TITLE
%----------------------------------------------------------------------------------------


%----------------------------------------------------------------------------------------

\begin{minipage}[t]{0.6\textwidth} % Start the left-hand side of the page
\vspace{0pt} % Trick for alignment


%----------------------------------------------------------------------------------------
%	WORK EXPERIENCE
%----------------------------------------------------------------------------------------

\section{实习经历}

%------------------------------------------------
% WORK EXPERIENCE 1
%------------------------------------------------
{\bf \large \textit{支付宝(中国)网络技术有限公司}} \hfill  \hfill {2013.7 至 2014.4\\}\\
{隶属于技术保障-数据库技术-OceanBase团队,实习期间,主要工作有两块:}
\begin{itemize} \itemsep -1pt 
\item obmonitor项目。obmonitor是为OceanBase数据库开发的监控告警平台,平台由我和另一个师兄两人维护开发,其中我重构了所有的告警模块,且在双十一期间为OceanBase开发了数据库监控大盘,系统能够稳定运行了线上。
\item log-center项目。日志中心项目是我个人独立为团队开发的一款工具,由于OceanBase是分布式数据库,出现故障需要DBA在多个组件间查找日志,而且OB每天产生的日志量也多达TB级别,所以需要一个自动故障检测和告警的工具。项目难点在于海量日志数据的近实时计算并对外提供查询服务,项目采用Storm实现实时计算,通过SolrCloud完成分布式索引和查询,web层采用Spring MVC+Spring+iBatis等框架。至我离职时,项目在线上运行有一个月。
\end{itemize}

%------------------------------------------------
% WORK EXPERIENCE 2
%------------------------------------------------
{\bf \large \textit{中淮供应链股份有限公司}} \hfill  \hfill {2014.6 至 2014.9\\}\\
{实习期间,主导开发了中淮供应链木材商城。主要工作包含:}
\begin{itemize} \itemsep -1pt
\item 团队三人,我负责主体架构,全程参与了设计与开发,完成了一个在线商城基本的前后台功能。项目难点在于对供应链需求的理解,技术难点主要是系统的性能和稳定性,主要解决手段是动静分离以及加缓存。由于这个项目主要由我负责,通过这个项目,加深了我对团队分工和团队合作的理解。
\item 项目站点:\href{http://www.zhchain.com}{中淮供应链木材商城}
\end{itemize}


%----------------------------------------------------------------------------------------
%	PROJECT EXPERIENCE
%----------------------------------------------------------------------------------------

\section{主要项目}

%------------------------------------------------
% WORK EXPERIENCE 1
%------------------------------------------------
{\bf \large \textit{algorithm-hodgepodge}} \hfill  \hfill {2014年\\}

\normalsize{C++项目。主要包含三个部分,一是实现了一个简易版的STL,包含主要数据结构、迭代器以及函数对象;二是典型算法问题的解答集合;三是一些排序算法的实现。这是一个杂糅的项目,个人兴趣项目,主要偏向数据将结构和算法的实现。}\\

{\bf \large \textit{火车票系统}} \hfill  \hfill {2013年\\}

\normalsize{Linux C程序。实现了简单的linux系统上的c/s架构,模拟铁路订票功能(简单功能)。项目重点在于IPC通信,数据存储采用mysql。}\\


{\bf \large \textit{Dessert House系统}} \hfill  \hfill {2013年\\}

\normalsize{J2EE项目。一个在线的甜品预定网站系统。主要模块包括会员功能模块(浏览、将商品加入购物车、预定等)、经理统计模块、总经理的店面管理,权限管理模块。后台使用了spring mvc + Hibernate开发。前台框架主要使用了Bootstrap和jQuery。}\\

{\bf \large \textit{汪星人购物搜索引擎开发}} \hfill  \hfill {2012年\\}

\normalsize{Java项目。一个购物垂直搜索引擎,实现了对主流商品网站的商品数据的抓取、信息提取、比价等功能。这是一个三人项目,我在项目中担当PM,既负责团队协作,也负责具体的开发工作。这个项目是我们参加中国软件杯的作品。获得了优秀奖。}\\

{\bf \large \textit{Forerunner轻博客系统}} \hfill  \hfill {2012年\\}

\normalsize{php项目。一个轻博客系统,功能类似于点点和lofter,具备发布文字、图片、视频、链接、音乐等功能。}\\


\end{minipage} % End the left-hand side of the page
\hfill
\begin{minipage}[t]{0.32\textwidth} % Start the right-hand side of the page
\vspace{0pt} % Trick for alignment

%----------------------------------------------------------------------------------------
%	COLORED BOX
%----------------------------------------------------------------------------------------
\colorbox{shade}{\textcolor{text1}{
\begin{tabular}{c|p{6cm}}
\raisebox{0pt} & 郝伟清 \\ % Education 
\raisebox{-3pt}{\Mobilefone} &  (+86)18751852080 \\ % Phone number
\raisebox{-1pt}{\Letter} & haoweiqinghaoson@gmail.com \\ % Email address
\raisebox{-4pt}{\textifsymbol{18}} & 江苏省南京市鼓楼区 汉口路22号 \\ % Address
\end{tabular}
}
}\\[10pt]


%----------------------------------------------------------------------------------------
%	SKILLS
%----------------------------------------------------------------------------------------

\section{主要技能} 

\begin{tabular}{rl}
& \\
语言
& C++, Java, Python, javascript, php\\
& \\
工具
& Vim, Git, SVN, CGDB, valgrind, \\
& SecureCRT, Eclipse, IntelliJ\\
& \\
系统
& Ubuntu, CentOS, Windows\\
& \\ \\
\end{tabular}

\section{教育经历} 

\begin{tabular}{rl}
& \\
2014.9 至 2016.6  & \textbf{南京大学软件学院硕士}\\
& \\
2010.9 至 2014.6  & \textbf{南京大学软件学院本科}\\ 
& \\
\end{tabular}\\[10pt]

%----------------------------------------------------------------------------------------
%	AWARDS
%----------------------------------------------------------------------------------------

\section{主要奖项} 

\begin{tabular}{rl}
& \\
2014  & \textbf{南京大学校级优秀毕业生}\\
& \\
2013  & \textbf{新鸿基奖学金}\\
& \\
2012	 & \textbf{第一届“中国软件杯”优秀奖}\\
& \textit{汪星人购物搜索引擎}\\ 
& \\
\end{tabular}\\[10pt]


%----------------------------------------------------------------------------------------
%	SCORE	
%----------------------------------------------------------------------------------------

\section{在校成绩} 

\begin{tabular}{rl}
& \\
\textsc{GPA}
& 4.3/5 \\
& \\
\textsc{排名}
& 全年级前20\% \\
& \\
\end{tabular}\\[10pt]

%----------------------------------------------------------------------------------------
%	LINK	
%----------------------------------------------------------------------------------------

\section{语言能力} 

\begin{tabular}{rl}
& \\
\textsc{英语}
& CET-6 \\
& \\
\end{tabular}\\[10pt]

\section{相关链接} 

\begin{tabular}{rl}
& \\
{GitHub:} &  \url{https://github.com/Haoson}\\
{个人博客:} & \url{http://haoson.github.io/}\\
& \\
\end{tabular}\\[10pt]

%----------------------------------------------------------------------------------------

\end{minipage} % End right-hand side of the page

\end{document}  